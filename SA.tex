Self-adaptive systems are systems that adapt based on the
environment \cite{macias2013self}. An underlying concept for self-adaptive systems is the MAPE-K loop \cite{kinneer2018managing}. Critical components to
the MAPE-K loop are Monitoring, Analysis, Planning, Execution and shared
Knowledge \cite{kephart2003vision}. The Monitoring component refers to
acquiring data from the managed system. This data will be processed and
used to update the content of the Knowledge component \cite{weyns2017software}. The
Analysis component will use the content in Knowledge to determine
whether an action/adaption is needed \cite{iglesia2015mape}. Any
action required will be used by the Plan component to create a plan.
This plan will then be executed by the Execute component \cite{kephart2003vision}. Figure \ref{fig:mapek} shows the MAPE-K process.
\begin{figure}
    \centering
    \includegraphics[scale=0.4]{images/mapek}
    \caption{MAPE-K loop }
    \label{fig:mapek}
\end{figure}
Using the MAPE-K process helps to ensure that the system adapts to
changes in the environment by monitoring the environment and managed
system, analysing the information collected, planning how to react and
then reacting.
\newline\newline
Underlying this MAPE-K loop are the concepts of an external and
internal principle. The external principle means a system that handles
changes in the environment, the system itself and the system's goals
\cite{weyns2017software}. The internal principle refers to the system being split
into two parts \cite{weyns2017software}:
\begin{itemize}
    \item the part interacting with the environment and concerned with the system goals, and 
    \item the part interacting with the first part and concerned with how to adapt to the system goals
\end{itemize}
In terms of the MAPE-K loop, each component relates to the internal
principle \cite{iglesia2015mape}. However, the entire MAPE-K loop
represents the external principle. This is because the whole loop
handles changes in the environment and system. Another view of a
self-adaptive system is the idea of self-management is set forth by
\cite{kephart2003vision}. This concept of self-management has four facets
\cite{kephart2003vision}:
\begin{itemize}
    \item self-configuration 
    \item self-optimisation
    \item self-healing 
    \item self-protection
\end{itemize} 
The combination of these four facets allows a system to adapt to its
environment. In the context of this research, a self-adaptive system
would be expected to minimise performance interference as the four
facets would result in the system aiming to avoid interference with
co-located tenants. Using self-adaption requires applying game theory
concepts because different tenants' actions affect resources
available for use \cite{glazier2017utility}. Otherwise, tenants may act
without considering the other co-located tenants which will worsen performance
interference.
