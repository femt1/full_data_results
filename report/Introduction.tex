Meeting Service Level Objective (SLO) contracts is a paramount concern for cloud service providers. Mitigating this concern is made difficult in multitenant cloud solutions because of the performance interference caused by soft resource usage limits for tenants. Performance interference also impacts the fundamental tenet of cloud computing: on-demand resources.
\newline\newline
Multitenancy is when there are multiple tenants, or users, in the same virtual environment who share resources \cite{dillon2010cloud}. Tenants sharing resources creates a problem, namely performance
interference, as the resources being used by one tenant reduce the available resources for other tenants \cite{mishra2013cloud}. In addition, if all tenants increase their resource usage, i.e. a
scenario with high load, there may then be insufficient resources
available to meet other requests. Cloud providers, as a result,  find it
challenging to meet SLOs as resources are constrained. Providers will then choose to prioritise particular tenants based on contracts, Service Level Agreements \cite{ru2014software}. 
\newline\newline
Inherently, cloud computing does not place any onus on tenants to
directly manage and curtail their resource usage  \cite{mell2011nist}. However, it is difficult for cloud providers to establish effective tools to monitor and adjust resources allocated to tenants on the fly \cite{gong2010press}.  Hence, a preferable solution is one where the tenants can indirectly manage resource usage without impacting performance. One such solution, for performance interference, is
managing resource-intensive background processes without seeing a
significant degradation in performance \cite{maas2014case}.  An example of a process, and the focus for this research, is Garbage Collection (GC) in Java.
\newline\newline
GC is responsible for memory management in managed languages, such as Java and Python. It is a naturally greedy and resource-intensive process that can actively consume CPU time, increasing the total resource utilisation of applications \cite{maas2016grail}. Patros et al.'s (2018) study highlights further that spikes in resource
utilisation are attributable to the GC. Implementing an elastic GC mode which adjusts the resources used by the garbage collector during runtime can manage these spikes and reduce the intensiveness of GC \cite{patros2018resource}.
\newline\newline
This research argues that the behaviour of the GC is similar to the
inherent greediness of a cat on a bed wanting more space. Therefore,
this research's solution is called \emph{A Cat On A Bed} and the three
different development phases are titled \emph{CatNap}, \emph{Circling}
and \emph{Cat's Meow}.
\newline\newline
These three development phases relate to the theoretical approaches
applied in the development process: naive,
\emph{Proportional-Integral-Derivative} (PID) controller-influenced and
\emph{Linear Quadratic Regulator} (LQR) approach. All three phases
develop a mode that adjusts the heap size, the number of GC threads and
the interval between regular collections. Applying a rigorous research
testing methodology ensures the quality of results. The end goal is to
develop a self-adaptive GC mode, part of the IBM OpenJ9 Java Virtual
Machine (OpenJ9), that elastically adjusts the three chosen variables to
reduce GC utilisation and hence, have a positive impact on performance
interference.
\newline\newline
In developing a realistic solution towards managing performance
interference through managing resource consumption of GC, there are two
research questions:

\begin{itemize}
\item
  Will an elastic GC mode that varies the heap size, number of threads
  and interval between standard GCs see a reduction in GC utilisation
  and hence overall CPU utilisation?
\item
  Is it possible to formulate Nash's equilibrium for co-located tenants
  using the elastic GC mode to reduce performance interference overall?
\end{itemize}
The former question discusses the impact of the solution developed
through this research on a single machine basis. The latter question
focuses on the impact of this solution on multitenant clouds. Also,
there are two hypotheses for the research questions:

\begin{itemize}
\item
  The proposed GC mode will reduce GC utilisation on the OpenJ9.
\item
  It will be possible to formulate a Nash's equilibrium\footnote{~A
    Nash's equilibrium is a scenario where no tenant is motivated to
    change their actions considering the other tenants' actions
    \cite{benslama2015game}. This will be discussed
    further in Chapter 2.} that will tend towards conservative GC to
  reduce performance interference overall.
\end{itemize}
The structure of this dissertation allows for the two research questions
to be answered. It will begin with an analysis of the relevant
literature including both related work and background literature.
Subsequent chapters will iteratively discuss the approach, design, and
testing of each development phase. A final discussion chapter will be
provided to summarise and analyse the findings from each development
phase. The final chapter will discuss future work and evaluate the
effectiveness of \emph{A Cat On A Bed} in addressing the two research
questions.
\subsection{Contribution}
This research makes the following contributions:
\begin{itemize}
\item
  A PID controller-driven elastic GC mode implemented on OpenJ9
\item
  An LQR controller-driven elastic GC mode implemented on OpenJ9
\item
  A game-theoretic model for two co-located tenants on a multitenant
  environment
\end{itemize}

