Cloud computing provides on-demand resources for clients. However, 
performance interference in multitenant scenarios can negatively
affect this provision of resources, hampering the ability to meet
Service Level Objective contracts. One controllable cause of performance
interference is resource-intensive background processes, such as Garbage
Collection (GC) in Java Virtual Machines. This research investigates the
impact of a less resource-intensiveness Garbage Collection (GC) on
performance interference through an elastic GC mode
called \emph{A Cat on a Bed}. The development of this mode applies three
approaches: naive threshold-based, \emph{Proportional-Integral-Derivative}
control theory and \emph{Linear Quadratic Regulator} control theory. Applying
these approaches to a single tenant highlights two points. Firstly,
control theoretic approaches can effectively manage GC; and secondly,
the same approaches perform better on larger-sized test benchmarks. In
particular, the \emph{Linear Quadratic Regulator} controller is more effective than
a \emph{Proportional-Integral-Derivative} controller in reducing GC; however,
there is a performance cost. Simulating multitenant scenarios using Kubernetes reiterates these findings and indicates that \emph{A Cat On A Bed} affects performance interference. In conclusion, this research identifies
that a control-theoretic motivated reduction in the resource-intensiveness of GC
generally has a positive impact on performance interference.
