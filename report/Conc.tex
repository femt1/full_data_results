
In approaching this research, there were two research questions:
\begin{itemize}
\item

  Will an elastic GC mode that varies the heap size, number of threads
  and interval between standard GC's see a reduction in GC utilisation
  and hence overall CPU utilisation?
  
\item

  Is it possible to formulate Nash's equilibrium for co-located tenants
  using the elastic GC mode to reduce performance interference overall?
  
\end{itemize}
Based on the three development phases, it is clear that it is possible
to manage and reduce GC utilisation both in single VM and multitenant
scenarios; thus, addressing the first research question. The development
phases also highlight that reducing GC utilisation can harm performance
time of applications. Generally, the naive threshold-based approach,
CatNap, was ineffective in managing GC utilisation. The other two
phases, Cat's Meow and Circling, were more effective than CatNap showing
that control-theory motivated approaches are appropriate for this
research problem. Furthermore, an LQR is better for GC utilisation
control but only with particular GC policies, OptAvgPause and
OptThruPut, and applications with high allocation rates or large number of
objects in a heap. The positive impact of the Cat's Meow JVMs, which
implemented an LQR controller, indicates that concurrent GC policies,
such as OptAvgPause and OptThruPut, work well with LQR controllers. The
PID controllers implemented in Circling showed better performance time,
meaning that the gain in other areas outweighs the cost of adding the
PID controller logic. However, Circling JVMs did not see a significantly
noticeable improvement in GC utilisation, meaning the benefits from the
PID controller must result in time gains in other areas in a JVM.
\newline\newline
The multitenant testing reiterated the findings from the single VM. It
further proved that the benefits of the different JVMs from Circling and
Cat's Meow are observable in multitenant scenarios. In addition, CPU
utilisation reduces for Circling JVMs, but it is less noticeable for
Cat's Meow JVMs.
\newline\newline
A game theory model was formed to address the second research question.
The model showed that Nash's equilibrium is met if both tenants do
nothing. However, this is not an ideal solution and indicates that there
needs to be an adjustment to how tenants view gains in cloud computing
to ensure tenants are motivated to reduce their resource usage.
\newline\newline
Therefore, based on the findings from this research, the initial
hypotheses were correct. These hypotheses were that it would be possible
to reduce GC utilisation and formulate Nash's equilibrium. However, the
reduction of GC utilisation primarily occurs with an LQR controller but
is dependent on the benchmarks and GC policies.

\subsection{Contribution}

This research makes the following contributions:

\begin{itemize}
\item
  A PID controller-driven elastic GC mode implemented on OpenJ9 that effectively reduces execution time of applications
\item
  An LQR controller-driven elastic GC mode implemented on OpenJ9 that better manages GC utilisation
\item
  A game-theoretic model for two co-located tenants on a multitenant
  environment
\end{itemize}

\subsection{Limitations}
One limitation of this research is the sole use of DaCapo benchmarks
\cite{blackburn2006dacapo}. Additional benchmark suites could
have been used to substantiate the results. Other benchmarks that were
considered include Renaissance \cite{prokopec2019renaissance} and
CloudSim \cite{calheiros2011virtual}. These benchmarks were not
used in this research due to the scope and timing of the research.
\newline\newline
Another limitation is the simplification of the GC model applied in
Cat's Meow. The model used simplifies the GC to being described by the
three chosen variables: heap size, number of GC threads and the interval
between local GCs. This is an over-simplification of the entire GC model
but was needed to provide proof-of-concept that LQR is an appropriate
approach for the research problem. Future work could focus on creating a
realistic GC model for the Eclipse OpenJ9 JVM.

\subsection{Future Work}
Potential future work from this research has already been referenced in
earlier parts of this dissertation. Generally, most future work focuses
on the further development of the existing models, i.e. LQR and PID. In
addition, a broader scope of testing suites could be used to evaluate
the different development phases better. A final area for future work is
evaluating the modes on realistic multitenant clouds to identify the
implications of adjusting GC resource utilisation.
