Game theory is used to mathematically model the interaction between independent parties \cite{Dubina2013}. Modelling the interaction between parties is useful in understanding how parties will interact
strategically to maximise their benefit \cite{brandenburger2002use}. Therefore, game theory is applicable in scenarios with multiple
players with different or conflicting motivations.
\newline\newline
In the application of game theory, there is an assumption that the
parties have their own goals they are trying to achieve \cite{Dubina2013}. The interactions include cooperation and conflict between the
players/parties \cite{bomze1986non}. Generally, there are two or more players
in order to have scenarios with conflict and cooperation. Underlying
game theory are some fundamental assumptions, these include:
\begin{itemize}
    \item  The players or parties are rational \cite{kaplinski2010game}.
    \item There are payoffs \cite{osborne1994course}
\end{itemize}
The first assumption, rational players, implies that they make
decisions or take actions that will bring them the most significant
benefit or minimise loss \cite{osborne1994course}. The second assumption about
payoffs implies that after all players have acted, there will be a
payoff. This payoff can be either positive or negative. A typical payoff
is monetary in nature \cite{owen2001empirical}.
\newline\newline
Based on these assumptions, applying game theory requires specifying
the following concepts \cite{owen2001empirical}:
\begin{itemize}
\item Players involved
\item Order of actions, i.e. who goes first and follows.
\item Possible actions of players 
\item Each player's knowledge about the previous actions taken by another player
\item Each player's knowledge about payoffs of all players
\end{itemize}
The player's set of actions comprise a strategy and incorporates
their knowledge of previous actions. The result of applying game theory
is to formulate an equilibrium. For game theory, an equilibrium is a
combination of strategies that ensure that each player's strategy is the
best strategy considering the other players \cite{liang2010studying}. Nash's equilibrium is one form of equilibrium.
\newline\newline
The application of game theory is usually limited to finite games
and players \cite{owen2001empirical}. The type of game is characterised by the type of information, how moves are made, if players
are working together and the number of stages. Type of information can refer to perfect or complete information.
\newline\newline
Firstly, perfect
information is when the player knows everything that the other players
have done \cite{huang2018resource}. Therefore, no information
is hidden. In contrast, imperfect information means that there is at
least one player who does not know the previous actions. 
\newline\newline
Secondly,
complete information means every player knows the potential payoffs for
everyone else \cite{guglielmi2018bayesian}.An example of this is the \emph{Prisoner's Dilemma}. A version of the
\emph{Prisoner's Dilemma} is provided below in the next part. In contrast, incomplete
information means that at least one of the players does not know the
potential payoffs for everyone \cite{liang2010studying}. An example of this
type of game is a Bayesian game. In a Bayesian game, there is incomplete
information for the players. However, they have some assumptions based
on known probabilities \cite{liu2006}. 
\newline\newline
Thirdly, a game
can also be simultaneous where the players either make their decisions
at the same time or unaware of the other player's actions while making
their own decisions \cite{savani2015game}. Fourthly, a game can
involve cooperation where the players work together to make decisions \cite{martinez2016formal}
\newline\newline
Finally, the number of stages can also determine the type of game.
If there are multiple stages or moves, the game is likely to be
Dynamic or Extensive \cite{owen2001empirical}. The stages can be infinite in size. Another
type of game is a Stochastic game where there is a start state \cite{liu2006}. States transition from each other based on a pre-calculated
probability. A Static or Strategic game is that there is only one move
or the players take actions at the same time \cite{Lye:2005:GSN:2701748.2701812}.
An example of a typical game theory scenario is the \emph{Prisoner's
Dilemma} provided below.
\newline\newline
\emph{Example: Prisoner's Dilemma}
\begin{displayquote}
`Two prisoners are accused of a crime. If one confesses and the other does not, the one who confesses will be released immediately and the other will spend 20 years in prison. If neither confesses, each will be held only a few months. If both confess, they will each be jailed 15 years. They cannot communicate with one another. Given that neither prisoner knows whether the other has confessed, it is in the self-interest of each to confess himself. Paradoxically, when each prisoner pursues his self-interest, both end up worse off than they would have been had they acted otherwise.' \cite{encylopediabritt}
\begin{figure} [hbt!]
    \centering
    \includegraphics[scale=0.75]{images/gametheor1}
    \caption{Payoff matrix for Prisoner's Dilemma}
    \label{fig:mgametheroy}
\end{figure}
\end{displayquote}
Based on the above scenario, the payoff matrix can be seen in Figure \ref{fig:mgametheroy}. From this payoff matrix, it is then possible to calculate the Nash's
equilibrium, which is both confessing. This represents an equilibrium where both prisoners are
aware of the punishments and the different options. In addition, there
is one better decision. For this scenario, the better decision is that
both the prisoners do not confess. Expected value functions can prove
which decision is better. In addition, both prisoners will end up with a
lower punishment if they confess. The better decision would change if
the prisoners did not make their decisions at the same time and were
able to know what the other prisoner was going to do.
\newline\newline
Even though game theory is usually applicable for scenarios
involving people, it is becoming particularly relevant for cloud
computing in multi-tenancy scenarios where the actions of a tenant can
affect other tenants.


