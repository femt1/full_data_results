The accepted concept of cloud computing is the provision of
computing resources on demand for tenants. Hence, tenants pay for
resources in a pay-as-you-go approach \cite{armbrust2010view}.
Therefore, tenants are not concerned with the management or provisioning of the resources \cite{nagesh2014resource}. On-demand resources is a crucial tenet of cloud computing,
and so the emphasis is on ensuring that tenants have access to all the computing resources
needed \cite{armbrust2010view}. Consequently, performance interference that
reduces the ability of co-located tenants to consume computing resources does
not align with this cornerstone of cloud computing. Co-located tenants and
their related impact on performance exists when there is multitenancy. Therefore, tenants share computing resources. This sharing creates issues when combined with soft resource limits as tenants can increase their resource usage even if they end up using resources set out for a
co-located tenant.
\newline\newline
There are three different classifications within cloud computing: 
\begin{itemize}
    \item Software as a Service (SaaS) that allows tenants to use the provider's applications over a
network \cite{furuncu2014scalablegame}
\item Provider as a Service (PaaS) that allows the
deployment of applications to a cloud \cite{farag2017research}
\item Infrastructure as a Service (IaaS) that provides infrastructure, such as
network or storage facilities, to tenants \cite{furuncu2014scalablegame}
\end{itemize}
\subsubsection{Service Level Objectives}
Another aspect of cloud computing is Service Level Objectives (SLOs). For cloud service
providers, SLOs are essential as they dictate what level of service is
necessary \cite{syed2017monitor}. Failing to meet SLOs can result in SLO
violations and affect the relationship with the tenant. Most SLOs are concerned with performance and can include metrics such as system uptime and I/O request reliability \cite{nathuji2010qclouds}. SLOs can also be related to reliability, flexibility and availability
\cite{mahdavi2017systematic}. In addition,
performance interference can result in SLO violations.
